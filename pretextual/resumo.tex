\begin{resumo}
%Elemento obrigatório, constituído de uma sequência de frases concisas e objetivas, em forma de texto.  Deve apresentar os objetivos, métodos empregados, resultados e conclusões.  O resumo deve ser redigido em parágrafo único, conter no máximo 500 palavras e ser seguido dos termos representativos do conteúdo do trabalho (palavras-chave).

Com a crescente demanda de alimentos e a diminuição da mão de obra no campo, o constante aumento da mecanização e da introdução da agricultura de precisão, os próximos passos da agricultura seguem em direção a automação das atividades trazendo a agricultura ainda mais perto das filosofias industrias, suprindo em vários aspectos   quanto a elevação dos níveis de produção, bem como a precisão na distribuição dos insumos agrícolas. O objetivo do trabalho aqui exposto é de criar um protótipo de veículo autônomo terrestre movido a energia solar que consiga, a partir de coordenadas e sensores, traçar uma rota que não passe pelo mesmo lugar mais de uma vez e que execute uma certa tarefa, evitando possíveis obstáculos presentes no campo, sendo capaz de recalcular sua rota e evitar colisões. Foram discutidos e apresentados cálculos referentes ao sistema fotovoltaico, um esboço do chassi do veículo, bem como uma breve discussão de como será controlado através do microcontrolador Arduino, além de um esboço do funcionamento do algoritmo de controle a ser produzido futuramente.

% TODO: se possível, escreva um resumo estruturado. Para TCC 1, o resumo estruturado teria os seguintes elementos:
% \textbf{Contexto:} \\
% \textbf{Objetivo:} \\
% \textbf{Método:} \\
% \textbf{Resultados esperados:} 
% ou, para TCC 2:
% \textbf{Contexto:} \\
% \textbf{Objetivo:} \\
% \textbf{Método:} \\
% \textbf{Resultados:} \\
% \textbf{Conclusões:}

% Palavras-chaves, separadas por ponto (tente não definir mais do que cinco)
\palavraschaves{Agricultura, Veiculo autônomo terrestre, fotovoltaico, Arduino}
\end{resumo}